% !TeX spellcheck = italian
\documentclass[12pt,italian]{report}
\usepackage{tesi}

\usepackage[a4paper]{geometry}		% Formato del foglio
\usepackage[italian]{babel}			% Supporto per l'italiano
\usepackage[utf8]{inputenc}			% Supporto per UTF-8
%\usepackage[a-1b]{pdfx}				% File conforme allo standard PDF-A (obbligatorio per la consegna)

\usepackage{graphicx}				% Funzioni avanzate per le immagini
\usepackage{hologo}					% Bibtex logo with \hologo{BibTeX}
\usepackage{epsfig}				    % Permette immagini in EPS
\usepackage{xcolor}                 % Gestione avanzata dei colori
\usepackage{amssymb,amsmath,amsthm} % Simboli matematici
\usepackage{listings}				% Scrittura di codice
\usepackage{url}					% Visualizza e rendere interattii gli URL
\usepackage[pdfa]{hyperref}			% Rende interattivi i collegamenti interni
\usepackage{tikz}                   % Permette di disegnare inline
\usepackage{import}
\usepackage{float}
\usepackage{xstring}                % Permette di usare la funzione \IfInteger
\usepackage{array}                  % Permette di usare colonne "m" dentro table
\usepackage{multirow}               % Permette di creare tabelle con righe multiple
\usepackage{caption}                % Roba per le figure side-by-side
\usepackage{subcaption}             % Roba per le figure side-by-side
\graphicspath{{immagini/}}

\usetikzlibrary{positioning}
\usetikzlibrary{calc}
\usetikzlibrary{shapes}

\usepackage[autostyle = false]{csquotes}
\MakeOuterQuote{"}

\newcommand{\GeneraSchemaPacchetto}[2]{{
\begin{tikzpicture}
    [
        box/.style={draw,rectangle,minimum width=\recminwidth, 
        minimum height=\rectangleheight, 
        outer sep=0pt, node distance=0pt}
    ]

    \def\rectangleheight{1cm}
    \def\recminwidth{2cm}

    \foreach \name/\size in {#1} {
        \node[box, label={below:\small \texttt{\size{} byte}}] (0) {\texttt{\name}};
    }    

    \foreach \name/\size [count=\i] in {#2}
    {
        \pgfmathtruncatemacro\prevposition{\i - 1}
        \node[box, right = 0pt of \prevposition,
            label={below:\small \texttt{\IfInteger{\size}{\size\ byte}{\size}}}
        ] (\i) {\texttt{\name}};
    }
\end{tikzpicture}
}}

\def\myCDL{Corso di Laurea in\\ Sicurezza dei Sistemi e delle Reti Informatiche}

% TITOLO TESI:
\def\myTitle{VirTEE: implementazione di un protocollo per la condivisione TEE in Cloud}

% AUTORE:
\def\myName{Massimo Perego}
\def\myMat{Matr. Nr. 965229}

% RELATORE E CORRELATORE: 
\def\myRefereeA{Prof. Andrea Lanzi}

% ANNO ACCADEMICO
\def\myYY{2022-2023}

% Il seguente comando introduce un elenco delle figure dopo l'indice (facoltativo)
%\figurespagetrue

% Il seguente comando introduce un elenco delle tabelle dopo l'indice (facoltativo)
%\tablespagetrue

\begin{document}

	\frontespizio
	\afterpreface

	\chapter*{Sommario}
	\addcontentsline{toc}{chapter}{Sommario}  
	\label{cap:sommario}
	
	\chapter{Introduzione}
	\label{sec:introduzione}
	
	\chapter{Trusted Execution Environment}
	\label{cap:TEE}
	Un \textit{Trusted Execution Environment} (TEE) è un'area isolata e sicura all'interno di un dispositivo, concepita con l'intento di operare in modo affidabile ed indipendente dal resto del sistema operativo. Questo permette di ospitare applicazioni e processi sensibili, eseguendo unicamente codice autenticato.
	
	I TEE stanno progressivamente guadagnando maggiore rilevanza in virtù del loro impiego per l'arricchimento di piattaforme preesistenti, offrendo soluzioni atte a incrementare la sicurezza e prevenire accessi o manipolazioni non autorizzate. Grazie a tali caratteristiche, i TEE trovano utilizzo in diverse applicazioni e ambiti operativi
	
	I TEE sono presenti sulla maggioranza dei cellulari tramite nomi quali \textit{TrustZone} di ARM, \textit{Samsung Knox}, \textit{Qualcomm QTEE}, \textit{Google Titan M} e \textit{Apple Secure Enclave}, con più finalità, tra le quali: autenticazione sicura, gestione di chiavi crittografiche e gestione dei diritti digitali.
	
	L'impiego dei TEE però sta assumendo un ruolo crescentemente significativo non solo nell'ambito dei dispositivi mobili, ma anche nei settori Desktop, Server e IoT. Sono sempre più numerosi i prodotti sul mercato che consentono esecuzione di codice sicuro anche in tali ambienti. Tale tendenza riflette la crescente preoccupazione per la sicurezza informatica e la protezione dei dati sensibili in un'ampia gamma di applicazioni e scenari operativi.
	
	In questo capitolo verranno esplorate le principali caratteristiche dei TEE e come tali implementazioni possano contribuire a migliorare l'affidabilità di vari tipi di sistemi.
	
	\newpage
	
	\section{TEE GlobalPlatform}
	\label{sec:GlobalPlatform}
	%Deve contenere: cos'è globalplatform, architettura, cosa ne consegue (che sicurezza e cosa forniscono)
	Il concetto di "Trusted Execution Environment" denota una combinazione di tecniche software e hardware progettate per creare un ambiente sicuro e protetto. Nonostante ciò, il termine risulta piuttosto generico, in quanto fa riferimento a una vasta gamma di prodotti con caratteristiche implementative e livelli di sicurezza diversi.
	
	L'assenza di standardizzazione potrebbe essere causa di confusione tra i consumatori e diventare un ostacolo per la portabilità ed interoperabilità di Trusted Applications (TA) su diverse implementazioni di TEE. Ciò implica la necessità di gestire una gamma di differenze nell'implementazione e garanzie di sicurezza offerte, complicando l'utilizzo di tali ambienti in modo uniforme e trasversale.
	
	A risoluzione di questi problemi il consorzio GlobalPlatform pubblica dei documenti contenenti specifiche che coprono vari aspetti chiave della realizzazione di TEE, favorendo l'interoperabilità tra dispositivi e sistemi.
	
	All'interno delle specifiche di GlobalPlatform, si possono trovare direttive dettagliate relative a diverse tematiche riguardanti i Trusted Execution Environments, tra le quali: il modello architetturale dei TEE, delineando le sue componenti fondamentali, un insieme standard di interfacce API e metodi per la gestione sicura delle credenziali.
	
	Un altro aspetto trattato è l'istituzione di parametri e criteri per la valutazione della sicurezza dei TEE. Attraverso queste direttive, si mira a garantire che tali ambienti protetti soddisfino precisi standard di robustezza e affidabilità.
	
	Un numero sempre crescente di organizzazioni sta adottando questi standard e adeguando le proprie soluzioni alle specifiche GlobalPlatform. La tendenza generale è di una sempre più ampia adozione di TEE in conformità con queste direttive, pertanto questo elaborato farà riferimento a TEE conformi GlobalPlatform.
	
	
	\bigbreak \noindent
	
	\subsection{Architettura TEE}
	\label{subsec:architettura}
 	Le direttive di GlobalPlatform non fanno riferimento ad implementazioni specifiche per quanto riguarda l'architettura, ma piuttosto delineano una struttura generale degli elementi logici costituenti il TEE e le loro interazioni.
 	
 	
 	\section{Applicazioni dei TEE}
 	\label{sec:applicazioni}
	
	\section{Supporto Hardware per i TEE}
	\label{sec:supporto-hw}
	%Da scrivere tutti i tipi di TEE (quelli scritti sopra, sgx, sev, trustzone, processori dedicati, TEE RISC-V) ed il relativo hardware
	
	\section{Vulnerabilità Comuni}
	\label{sec:vulnerabilità}
	
	
	
	\chapter{TEE in Ambiente Cloud}
	\label{cap:problema}
	%Virttee design problems, problema nel portare roba in cloud e come risolverli
	%Problematiche, come sono state affrontate
	%Componenti senza implementazione, mettere schemi di funzionamento
	
	
	
	\chapter{Sicurezza dei Dati}
	\label{cap:implementazione}
	%Dettagli riguardo all'implementazione, soluzioni tecniche trovate
	%Iniziare con ciò che ha fatto marco? 
	%Section del capitolo prima?
	
	
	
	\chapter{Analisi Performance}
	\label{cap:dati}
	%Riportare performance generali del sistema, dati di Marco
	%Analisi performance con l'implementazione
	%Commentare i risultati, perché alcuni sono 4% altri 150%
	
	\chapter{Conclusioni}
	\label{cap:conclusioni}
	%Lavori futuri e conclusioni
	
	\bibliographystyle{unsrt}
	\bibliography{bibliografia}
	\addcontentsline{toc}{chapter}{Bibliografia}

\end{document}

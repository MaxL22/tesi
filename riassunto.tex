\documentclass[12pt,italian]{report}
\usepackage{tesi}

\usepackage[a4paper]{geometry}		% Formato del foglio
\usepackage[italian]{babel}			% Supporto per l'italiano
\usepackage[utf8]{inputenc}			% Supporto per UTF-8
\usepackage[a-1b]{pdfx}				% File conforme allo standard PDF-A (obbligatorio per la consegna)
\usepackage{eso-pic}

\usepackage{graphicx}				% Funzioni avanzate per le immagini
\usepackage{hologo}					% Bibtex logo with \hologo{BibTeX}
\usepackage{epsfig}				% Permette immagini in EPS
\usepackage{xcolor}				% Gestione avanzata dei colori
\usepackage{amssymb,amsmath,amsthm} % Simboli matematici
\usepackage{listings}				% Scrittura di codice
\usepackage{url}					% Visualizza e rendere interattii gli URL
\usepackage{hyperref}				% Rende interattivi i collegamenti interni
\usepackage{tikz}                   % Permette di disegnare inline
\usepackage{import}
\usepackage{float}
\usepackage{xstring}                % Permette di usare la funzione \IfInteger
\usepackage{array}                  % Permette di usare colonne "m" dentro table
\usepackage{riassunto}              % Personalizzazioni per il riassunto
\graphicspath{{immagini/}}

\usepackage[autostyle = false]{csquotes}
\MakeOuterQuote{"}

\begin{document}
	
	\chapter*{Riassunto elaborato}
	
	\AddToShipoutPicture*
	{\put(450,593){\includegraphics[width=23mm,height=23mm]{immagini/unimi}}}
	
	\textbf{Titolo elaborato:} VirTEE: implementazione di un protocollo per la condivisione TEE in Cloud
	
	\vskip 0.8 cm
	
	Nell'era digitale, la crescente importanza della sicurezza delle informazioni e della protezione dei dati ha posto l'attenzione su nuove sfide e soluzioni innovative per garantire la confidenzialità e l'integrità delle informazioni sensibili. Tra le molteplici tecnologie e approcci sviluppati per affrontare tali sfide, i \textit{Trusted Execution Environments} (TEE) emergono come uno strumento importante per la sicurezza informatica.
	
	I TEE sono "ambienti di esecuzione che funzionano parallelamente, ma in modo isolato, dal sistema operativo principale del sistema" \cite{gp2020systemarchitecture}. Permettono di eseguire software con requisiti di sicurezza particolarmente elevati in un ambiente separato dalla classica gerarchia di privilegio del sistema, tramite una base software e hardware separata e di dimensione ridotta, tendendo a minimizzare la superficie di attacco per possibili minacce.
	
	\medbreak
	
	Questi sono stati sviluppati principalmente con in mente gli ambienti mobili e ad oggi sono ampiamente utilizzati in questo ambito, ad esempio per la gestione delle chiavi crittografiche \cite{androidkeystore} o per la protezione dei contenuti multimediali \cite{widevine} \cite{playready}.
	
	Con la sempre crescente rilevanza del cloud computing si è cominciato ad avere interesse nell'utilizzo dei TEE anche in questo ambito \cite{demigha2021hardware} \cite{dai2010tee}, in quanto si tratta di soluzioni di sicurezza imposte tramite hardware e di conseguenza trasparenti anche per i clienti riguardo alle garanzie e rischi che ne conseguono.
	
	\medbreak
	
	Le soluzioni generalmente considerate per l'utilizzo di TEE in cloud prevedono che l'utente debba portare il suo TEE, con le complicazioni dovute alla gestione dello stesso. 
	
	Per evitare le complessità di gestione si è cercato un modello alternativo che permetta al cloud service provider di fornire il TEE come parte dei servizi \cite{tesi_cutecchia}, senza essere in contrasto con il modello precedente ma offrendo un'alternativa. 
	
	Questo metodo però porta a nuovi dubbi riguardo soprattutto la confidenzialità dei dati. Il fornitore del servizio rimane intermediario tra macchine virtuali ospitate e TEE, quindi con la possibilità di accedere ai dati scambiati. Tradizionalmente, per natura del servizio stesso, il fornitore è sempre stato considerato un ente fidato ma questo concetto non vale più per dati che richiedono sicurezza elevata, come potrebbero essere quelli trattati da un TEE. 
	
	\medbreak
	
	L'obiettivo di questo elaborato era presentare una possibile implementazione che permette di risolvere il problema riguardante confidenzialità ed integrità durante lo scambio dei dati tra TEE e sistema ospite.
	
	Per fare ciò è stato necessario implementare un protocollo crittografico ed un algoritmo per lo scambio chiavi tra la macchina virtuale gestita dal cliente ed il Trusted Execution Environment. 
	
	Sono state inoltre verificate le performance di questo sistema, le quali risultano accettabili, anche se non nella totalità delle applicazioni pratiche.
	
	\medbreak
	
	Questa però non è un'implementazione definitiva, in quanto per un'integrazione totale di tale modello in ambiente cloud rimangono alcuni problemi da risolvere. 
	
	Ad esempio, in questo elaborato non è stato trattato il problema di memorizzazione delle chiavi crittografiche, che sarebbero visibili service provider. Inoltre sono possibili miglioramenti anche dal punto di vista di trasparenza ed usabilità da parte dell'utente finale.
	
	Per concludere, questo lavoro può essere considerato un ulteriore passo aventi nella direzione un'implementazione completa.
	
	
	\bibliographystyle{unsrt}
	\bibliography{bibliografia}
	\addcontentsline{toc}{chapter}{Bibliografia}
	
\end{document}